\chapter{引言}\label{chap:introduction}

随着计算机技术和网络的迅猛发展,信息技术已经深刻改变社会的管理方式和人们的生活方式。可以说信息已经成为国家、企业和个人的重要资产,因此,保证信息的安全对于个人、企业乃至国家战略都是至关重要的。然而在利益的驱使下,针对计算机平台和网络的攻击层出不穷,使得政府、企业和个人都疲于应对。传统的信息安全保障手段主要以防火墙、入侵检测和病毒防范为主,都是以保护服务器和网络设备为主,而不重视终端的保护。而且防护的手段主要是采取封堵的办法,即捕捉滞后的特征信息,不能科学预测可能的攻击和入侵,难以抵御持续变化和迅速增长的威胁和攻击。因此,有必要从底层采取措辞来提高计算机的安全防护能力,从源头上解决人与程序、人与机器还有人与人之间的信任问题,从而有效解决信息系统的安全威胁。

可信计算的思想就是由此产生,通过在硬件层面引入安全模块,基于密码技术建立信任根、安全存储和信任链机制建立计算机系统的整体安全。经过多年的发展,可信计算技术已经不仅仅局限于传统的计算平台和环境,在移动、虚拟等平台以及与云计算、物联网等计算环境都体现出巨大的应用价值。特别是,新一代可信计算基础标准TPM 2.0在体系结构上与前代相比进行了革新,明确了其对移动和虚拟话平台的支持。由于新版TPM 2.0标准安全性并未得到充分认证,对可信计算技术的理论分析研究和推广应用也就重新燃起需求。

\section{研究背景}\label{sec:background}

信息技术的飞速发展使得各类计算平台已经深入应用到社会、经济、军事的各个领域。广大用户在享受技术进步带来便利的同时,也面临着各类安全事件所带来的危害,如恶意代码植入,计算机病毒感染,漏洞频出,隐私数据被窃取。从国家层面来说,这类问题更为严峻,涉及国家机密和国家安全。因此,如何保证终端平台的安全是信息安全领域一个重要课题。传统的安全解决方案或采用纯软件的形态,或采用专用的硬件设备。前者很容易受到计算平台内运行的其他软件或网络通信的影响,而后者又受限于高昂的造价和不同产品的异构型限制。因此有必要引入一种新型的安全硬件,以此为基础构建安全方案,即保证其经济性,又能保持其统一性和不同厂商产品间的互操作性。

可信计算思想就是由此产生,通过在应硬件层面引入安全模块,基于密码技术建立信任根、安全存储和信任链机制。由此建立起来的可信计算组织(TCG:Trusted Computing Group,前身为可信计算联盟TCPA:Trusted Computing Platform Alliance)自2001年起发布了一系列可信计算相关规范,其中以PC和服务器为主要应用环境的可信平台模块规范TPM 1.2 \citep{TPM1},详细规定了上述硬件“安全模块”的功能、软硬件接口、安全特性和实现方式。这个规范已成为了ISO国际标准\citep{ISO1}。为了提高平台和设备安全性,以应对当下多变的平台环境以及安全需求,TCG组织于2013年正式发布TPM 2.0版本草案,并于2014年10月30日发布TPM 2.0 正式版\citep{TPM2},ISO/IEC组织随后于2015年将TPM规范版本更新至TPM 2.0正式版\citep{ISO2}, 标志着可信平台模块正式进入2.0时代。

TPM 2.0版本规范的发布也使得产业界快速向TPM 2.0应用迁徙,得到了普遍应用。TPM安全芯片已经被广泛部署到PC机,笔记本电脑以及服务器上,芯片厂商和操作系统提供商也都明确表示对TPM 2.0的支持,比如Intel Skylake平台已经全面支持TPM 2.0应用;微软Window 10操作系统也加强了可信计算技术的应用,充分利用TPM进行访问控制和身份认证;Linux Kernel 4.0起也开始支持TPM 2.0驱动;IBM也已经发布了TSS 2.0中间件。国内各大硬件厂商(比如联想和国民技术等)也对TPM 2.0的发展与推广贡献了重要核心作用,比如TPM 2.0新增了对国产密码学算法的支持;同时吸收和学习了中国TCM(Trusted Cryptographic Module)技术中广泛使用对称密码的方式以提高应用性能。

TPM 2.0与TPM 1.2相比,从算法灵活性、功能多样性、平台安全性和效率等方面,有了显著的提升。比如增加了对国产算法的支持,新增了策略授权功能,加强了算法的强度以及改进了授权会话的安全性,采用更加高效的基于椭圆曲线的直接匿名证明方案。这些新增和加强的功能说明TPM 2.0 不仅仅是从TPM 1.2单纯的继承过来。因此,之前针对TPM 1.2平台功能的安全性分析于研究并不能直接说明TPM 2.0的平台安全性,需要重新针对TPM 2.0协议和接口的安全性进行评估,以此反应新增和增强的功能是否还符合所期望定义的安全。幸运的是,虽然可信计算基础理论研究相对滞后,但是经过TPM 1.2的研究与发展,已经具有了一些经典的分析方式,其中形式化分析便是常用手段之一。其威力已经在TPM 1.2的研究与发展中得到充分体现,发现了许多漏洞与威胁,对TPM的发展立下了汗马功劳。因此,本文也是充分利用形式化分析方法对TPM 2.0的协议与协议进行研究,以期得到平台安全性的理论保障。

除了保证平台安全外,TPM 2.0对隐私的保护的要求也在提高。 TPM在进行远程证明的时候,需要想远程验证方证明自己是一个值得信赖的通信实体,即进行平台身份证明。最初TCG采用可信第三方PrivacyCA协助TPM完成身份证明,从而避免平台身份信息的直接泄漏,但是PrivacyCA是完全知道TPM的真是身份,这也在一定程度上存在隐患。基于此种缺陷,TCG提出了直接匿名证明(DAA:Direct Anonymity Attestation)的TPM身份认证方法,使得平台拥有者能够直接向远程验证方证明自己的真实身份,同时又不暴露自己真实身份信息。TPM 1.2采用了基于RSA算法的DAA协议(以下简称RSA-DAA),TPM 2.0采用了基于椭圆曲线双线性对的DAA协议(以下简称ECC-DAA)。ECDAA方案有着比RSADAA方案更加高效的运行效率,同时由于TPM 1.2中对DAA协议接口定义的高度封装性,使得其使用灵活度不高。而TPM 2.0中对实现DAA协议的接口定义灵活,不仅能够实现不同双线性对方案的ECC-DAA协议,甚至可以实现与DAA相同思想的匿名凭证类协议,使得TPM 2.0的应用更加广泛。本文的研究也在DAA协议的基础上,研究了其他类型的匿名凭证类协议。

%除了保证平台安全之外,TPM 2.0对隐私的保护也逐渐加强。TPM在进行远程证明的时候,需要向远程验证方证明自己是一个值得信赖的通讯实体,即进行平台身份证明。如何直接采用标识TPM身份的背书密钥证书来进行远程证明,则显然会泄露平台的身份证书。因此TCG采用可信第三方PrivacyCA的方法协助TPM完成身份证明,从而避免平台身份信息的直接泄露。由于它与原有的PKI体系一脉相承,部署方便,因此很多基于安全芯片的解决方案都采用这种方法进行平台身份证明。然而尽管PrivacyCA提供了一定程度的平台身份信息隐私保护,但是由于PrivacyCA完全知道TPM的真实身份,如果验证方与PrivacyCA两者串谋,TPM的身份隐私是没有保障的。基于PrivacyCA身份认证的此种缺陷,TCG提出了直接匿名证明(DAA)的TPM身份认证方法,并在TPM的技术规范中支持DAA。使得平台拥有者能够直接向远程验证方证明自己的真实身份,同时又不暴露自己真实身份信息。TPM 1.2仅支持基于RSA的DAA协议,TPM 2.0开始采用更加高效的基于椭圆曲双线性对的DAA协议。

%DAA协议在平台远程证明以及可信网络中扮演者重要的作用。只有一个可信的平台证实了自己的真实身份,才被允许进行下一步的操作(远程证明或者可信网络接入)。TPM 2.0采用了比TPM 1.2中RSA-DAA更加高效的ECDAA方案,然而还是没有明确定义如何进行证书的撤销,即如果平台被敌手攻克,如何对其身份进行曝光或撤销。不仅如此,在DAA协议的基础上,还研究其他与DAA相同思想的匿名凭证类协议,从而能够全面研究在可信网络中的隐私和安全问题。

%综上所述,本课题将针对可信计算中安全性和隐私性进行深入研究。在安全性研究方面,将研究可信计算安全性分析理论和方法,对可信计算协议和接口进行有效的安全性分析,并尝试改进现有分析方法;同时利用安全性分析结果,结合具体的应用场景,进行安全性评估,改进应用场景的安全性。在隐私性方面的研究,以DAA协议为基础,深入研究匿名凭证类协议和系统,利用现有的无可信第三方(TTP)匿名凭证类系统的撤销功能,研究与改进TPM中 DAA的撤销功能。同时可以利用TPM安全芯片来实现匿名凭证类系统,提高其安全性。最后可以对匿名凭证类系统进行形式化安全分析。

综上所述,本课题将针对新一代可信计算技术的安全性和隐私性进行深入研究。在安全性研究方面,利用形式化分析方法对装载TPM 2.0技术的可信计算安全芯片提供的授权会话、密码功能支撑命令等接口的安全性进行形式化分析;在隐私性方面的研究,在深入了解DAA协议的基础上,研究无可信第三方的匿名凭证类协议和系统。特别是这类协议所具有的匿名黑名单功能,可以促进DAA协议的应用延伸。反之,针对DAA协议的形式化分析方法,亦可以进行改进与发展,使之适应于含有匿名黑名单机制的匿名凭证协议,发现其漏洞并加以改进。最后,基于TPM 2.0的接口,我们还能实现这类匿名凭证协议。本课题的研究意义在于:

(1)无论是国外还是国内,在可信计算领域都处于技术超前于理论,理论滞后于技术的状况。可信计算的理论研究落后于技术的开发。进行可信计算安全性分析理论和方法的研究,可以推进可信计算理论研究的发展。

(2)对可信计算协议与接口进行安全性分析,结合其应用场景,给出安全评估,改进在应用场景的安全性,有利于可信计算技术的推广与应用。

(3)对匿名凭证类系统的研究,有利用DAA协议的发展与应用。

(4)结合我国信息安全管理的国情,推进可信计算芯片的本土化进程,为我国新一代可信计算的设计与分析提供借鉴。

\section{研究现状} \label{sec:status}

本文研究基于TPM 2.0的协议,可以粗略的分为两个层面的协议:一是基于芯片的基础接口组合协议,定义了可信计算安全芯片的安全机制,例如密钥管理、授权认证、密码学支撑服务(加密、签名等)等接口组合协议。二是基于调用这些基础接口实现的更为复杂的协议,例如直接匿名证明协议和匿名凭证方案等。本节分三个维度介绍课题相关的研究现状,分别是安全协议形式化分析方法、可信计算基础协议和接口的安全性分析以及可信计算延伸协议的设计与分析。

\subsection{安全协议形式化分析方法研究现状}

形式化分析方法在开发一些注重安全和保障的关键系统上,逐渐地引起广泛的关注。在安全协议这个领域尤其得到了体现。一些重要的安全协议,比如安全传输层协议TLS以及它的前任安全套接层协议SSL,出现细微的缺陷都会导致巨大的经济损失。因此在设计协议的时候保证其安全性显得尤为重要。而形式化方法就是对安全协议的安全性进行分析验证,从而克服人为证明的易错性。

一般公认,\citet{needham1978using}最早提出了安全协议形式化分析的思想,但其主要工作仅仅是建立了Needham-Schroeder协议。而第一个用形式化方法对协议进行分析的文章,当属\citet{dolev1981security}在1981年的结果,建立了验证协议安全性的Dolev-Yao模型。他们用算法的手段分析了两类特殊的协议的安全性。随后的\citet{dolev1983security}的分析方法与之一脉相承,都是用多项式时间的算法对于某些特定类型协议的安全性进行判定。此外,基于Dolev-Yao模型的模型检测工具也相继开发出来,例如Interrogator\citep{millen1987interrogator}和NRL协议分析器\citep{meadows1994model}等。

Dolev-Yao模型提出之后,形式化方法又一里程碑的工作是由\citet{burrows1989logic}提出的BAN逻辑,他们利用知识和信念逻辑,描述和推理认证协议。沿着这个方法,许多逻辑被构造了出来,其中大部分是BAN逻辑的变种。BAN逻辑之后出现的较为新颖的定理证明技术,包括了Paulson归纳法\citep{paulson1998inductive}、串空间\citep{fabrega1998strand}、类型检测\citep{abadi1999secrecy, gordon2003authenticity}和其他方法\citep{kemmerer1994three,dutertre1997using}等。根据这类定理证明技术开发的定理证明器如PVS\citep{owre1992pvs}和Athena\citep{song2001athena}等。

在上述这些基于Dolev-Yao模型(以下简称符号模型)的形式化分析方法和工具蓬勃发展时机,另外一种基于计算复杂性模型(以下简称计算模型)的形式化方法也悄然形成。一种为间接方式,即通过证明基于Delve-Yao模型的形式化方法的计算可靠性,从而得到形式化分析的可证明安全结论。另一种为直接方式,即基于游戏序列的证明技术,通过进程演算语言描述协议,从而直接得到协议的可证明安全结论。

因此,从敌手模型类型进行分类,可以将形式化分析分为符号模型下的形式化方法和计算模型下的形式化方法。由于形式化分析方法呈现“百家争鸣”的景象,本课题着重研究了两种敌手模型下部分的形式化分析方法。下面介绍的形式化分析方法的发展现状是与本课题研究内容相关的:基于符号模型的形式化方法和基于计算模型的形式化方法。

\subsubsection{基于符号模型的形式化分析方法现状}

基于标准Pi演算语言\citep{milner1999communicating},\citet{abadi2001mobile,abadi2017applied} 于2001年提出了应用Pi演算的概念,与\citet{abadi1997calculus}提出的spi演算的主要不同之处体现在对密码原语的操作上,spi演算内建固定原语,而应用Pi演算利用等价理论,可以支持自定义的更多复杂的原语。此外,应用Pi演算的出现也促进了自动化分析工具的发展。

2001年,\citet{blanchet2001efficient}提出了一个基于Prolog规则的协议分析器,用于分析协议的机密属性。协议由Prolog规则表示,并且用一个高效的算法确定某一事实是否能从这些规则(知识)中推导出来。2002年,\citet{abadi2002analyzing,abadi2005analyzing}开发了两个协议分析技术,一个基于应用Pi演算的类型检测进程语言,一个基于无类型的逻辑程序(Prolog)。并且证明了两种技术的等价性。从而给出了从应用Pi演算建模的进程语言到逻辑语言的转化,提出了协议自动验证工具Proverif。Proverif工具将协议用基于应用Pi演算的进程语言表述,然后转化为Horn字句的表示,最后利用这些Horn语句(包括协议和敌手能力等),用可达性理论来求解某一个事实是否可达(证明机密属性)。与模型检测方法相比,Proverif没有了状态空间爆炸的问题。然而,Proverif不是完备的,它会产生错误攻击。但是实验表明错误攻击极少出现,分析成功率高,因此该工具得到了广泛应用,例如\citep{blanchet2008automated,bhargavan2017verified}。

随后,基于应用Pi演算和Proverif的研究工作越来越多。其中,Blanchet等人继续扩展ProVerif支持的安全属性,包括对应性(可证明协议的人这个属性)\citep{blanchet2002secrecy,blanchet2009automatic}、强机密性\citep{blanchet2004automatic}、选择项等价\citep{blanchet2005automated}。然而选择项等价的假设是强于观察等价的,因此只适用于证明一类等价关系,即两个进程为同一进程的不同版本,仅选择的项不同。为了使ProVerif支持更多的等价关系,\citet{cheval2013proving}通过定义项的重写规则,使之可以支持含有条件分支的等价关系。\citet{blanchet2016automated}也拓展了选择项等价,使之可以证明需同步的两个进程间的等价关系。

还有一部分的工作集中在对应用Pi演算和ProVerif的扩展上,\citet{arapinis2011statverif}在ProVerif的基础上,加入全局状态补丁,使之可以分析代全局状态的协议,并且命名为Statverif;之后他们同步跟进了对应用Pi演算理论的全局状态扩展\citep{arapinis2014stateful}。另外一个基于应用Pi演算进行全局状态扩展是\citet{kremer2016automated}开发的分析有限状态协议的SAPIC工具,该工具的协议表达为基于应用Pi演算的进程语言,然后转化为多重集重写规则,再借助Tamarin工具\citep{schmidt2014automated}进行分析。同时,作者证明了工具的完备性。

\citet{paiola2012verification}提出了一种使用有限长度列表建模和分析无限长度列表的协议的方法,然而他们协议建模语言是广义上的Horn子句,还不能用ProVeif语言建模,因还不支持ProVerif进行自动分析。\citet{chothia2015automatically}提出在进程中插入“阶段”,避免一类由ProVerif分析私密性而产生的错误攻击,比如分析承诺协议中的私密性。目前ProVerif仅实现了在非复制进程中插入“阶段”进行分析。

\subsubsection{基于计算模型的形式化分析方法现状}

自从\citet{abadi2000reconciling}提出在某种情况下,Dolev-Yao模型下的安全性定义蕴含计算模型下的安全性定义,许多工作开始热衷于关联两种模型的研究。这些工作的结果给出了在Dolev-Yao模型下的计算可靠性,例如\citep{backes2003composable,cortier2005computationally,janvier2005completing}。这些工作大都没有实现自动化证明。之后出现利用Dolev-Yao模型的自动化工具证明协议的安全性,再利用定理证明其计算可靠性,例如\citep{canetti2006universally}。一个比较系统的工作是由\citet{backes2009cosp}提出的通用证明框架CoSP(Computatioanl Soundness Proofs),该框架能够在符号模型下进行计算可靠性证明,将密码学原语证明和形式化演算分离开进行模块化证明。因此CoSP框架特别适合证明形式化演算类工具的计算可靠性证明。基于CoSP框架,其作者进行了一系列的后续研究工作,例如利用该通用框架证明了形式化分析工具F7\citep{backes2010computationally}和ProVerif\citep{backes2014computational}的计算可靠性结果。\citet{shao2016computational}也利用此框架证明了带状态的应用Pi演算的计算可靠性。

上述这些研究都是属于采用间接方式的到计算可靠性结果。由于符号模型和计算模型不可能精确对应,有的为了其证明方法的可靠性需要额外的假设,其因此这类的结果总有其限制性。基于此,学者开始考虑直接利用形式化方法来证明协议的计算可靠性,从而绕过符号模型。2005年,\citet{halevi2005plausible}讨论了利用游戏序列证明技术实现自动化证明工具的可行性。2006年,\citet{bellare2006security}实现基于代码的游戏序列证明框架,并且利用该框架证明了3-DES对称加密算法。

一个比较系统的工作是由\citet{blanchet2006computationally,blanchet2008computationally}设计和实现的在计算模型下的全自动证明工具Cryptoverif,该工具利用概率多项式的进程演算语言来表示协议,利用游戏序列组织证明协议的机密属性。翌年,Blanchet扩展Cryptoverif\citep{blanchet2007computationally},使其支持证明认证属性。该工具已经成功证明许多协议,比如Yahalom协议、Needham-Schroeder协议、Kerberos协议\citep{BlanchetJaggardScedrovTsayAsiaCCS08}、SSH协议\citep{cade2013computationally}和TLS 1.3协议\citep{BlanchetCSF18}。

另外一个系统的工作是由\citet{barthe2011computer}设计的在计算模型下自动化验证工具EasyCryp(其前身CertiCrypt\citep{barthe2009formal}),该工具也是基于游戏序列证明协议,其提供的自动化证明方式为从一个协议的证明骨架去自动化补全或者说详细阐述为一个完整的Game序列证明过程。因此,其与Cryptoverif相比更注重验证性,即EasyCrypt需要使用者先提供证明的框架,然后它进行完善及验证该框架的正确性。

Blanchet团队还有一个工作计划是将ProVerif和CryptoVerif的输入脚本语言统一,这样一份协议的输入脚本可以同时验证其在符号模型下的安全性,也可以得到在模型下的可安全性证明,达到优势互补的目的。该项工作目前取得了突破进展,在最新发布的ProVerif 2.0和CryptoVerif 2.0已经实现大部分输入语言的兼容。这方面典型的分析工作是ARINC823公钥和分享密钥协议\citep{blanchet2017symbolic}。这两个协议是航空协议,用于空对地交流。

\subsection{可信计算基础协议和接口的安全性分析现状}

在过去的几年中,许多可信计算基础协议和接口(全称为应用程序接口,Application Programming Interface,以下描述有时也称做API接口)的缺陷被发现,这些问题大都与机密性和认证性相关。在发现攻击和漏洞的过程中,形式化分析方法扮演了重要的角色。而且随着形式化方法的发展,匿名性和其他各类高级安全属性也逐渐被描述出来。因此形式化分析方法的应用范围也更加广阔,包括应用于新一代可信计算平台TPM 2.0协议和接口的分析中。下面就近年来可信计算平台(包括TPM 1.2等早期版本)基础协议和接口分析现状做一个简要的概述。

\citet{lin2005automated}利用定理证明器Otter和Alloy分析了TPM 1.2的大量接口组合,他的分析结果包括发现授权协议中的密钥句柄转换攻击等。\citet{bruschi2005replay}证明了OIAP协议易受重放攻击,攻击者可以重发已执行的命令。\citet{gurgens2007security}利用有限状态自动机分析了TPM 1.2的API接口组合,发现在远程证明的证书申请过程中,攻击者可以非法获得其选择密钥的证书。\citet{datta2009logic}设计了一种安全系统逻辑,并分析了TPM完整性收集和报告功能,发现攻击者可以任意修改PCR值,破坏信任链传递机制。\citet{chen2009offline}发现了一个对授权数据的离线字典攻击,而后又给出了共享授权数据的应用场景下中间人攻击的问题\citep{chen2009attack}。\citet{delaune2010formal}利用Proverif工具分析了TPM 1.2部分API接口,但是忽略了PCR状态。随后他们又分析了基于PCR状态的认证协议建模方法\citep{delaune2011formal},扩展了Proverif分析范围。

随着TCG组织发布TPM 2.0标准,一些基于TPM 2.0协议和API的分析也随后出现。\citet{shao2013type}利用类型检测系统分析了TPM 2.0的存储保护部分。\citet{zhang2014formal}利用Tamarin工具分析了TPM 2.0密钥管理部分接口,发现并修正了在TPM 2.0中密钥迁徙过程中存在的攻击。\citet{zhao2015security}分析TPM 2.0中SM2密钥交换协议,发现由于TPM 2.0的接口设计,其还是存在未知密钥攻击和密钥泄漏伪装攻击,并提出修补方案。\cite{Shao2018Formal}利用有状态的应用Pi演算对基于HMAC的授权协议进行建模,并利用Tamarin工具分析发现在TPM初始化后缺少安全会话的攻击场景存在中间人攻击,导致TPM的调用者不能建立安全会话,并提出了修补方案。

可以看出,对应用程序接口组合的分析都是利用形式化模型下的分析工具或者分析方法,致力于寻找其可能存在的缺点和攻击。还有一个研究路线是致力于应用程序接口框架的可证明安全。\citet{Cortier2009A,Cortier2014A}提出一个对称密钥管理API接口框架,并且定义了一个形式化安全策略,手动证明其安全性。\citet{Daubignard2014A}应用其可证明方法扩展了该API接口使得支持非对称密钥管理。\citet{Cachin2009A}提出一个可证明安全的密码学API接口,他们利用现代密码学定义了接口的安全策略并手动证明了接口的安全性。\citet{ChuF15}利用他们的方法,用现代密码学定义了TPM 2.0中密码学接口的安全策略并证明了安全性。

\subsection{可信计算延伸协议的设计与分析现状}

匿名凭证系统(anonymous credentials, 也称为PABC,即Privacy-enhancing attribute-based credentials)是一些注重隐私的认证系统的核心组成部分,允许用户盲申请证书(即不暴露或暴露部分属性),同时能够以不可链接的方式证明自己拥有该证书且不暴露隐藏的属性。一般认为第一个提出匿名凭证思想的是\citet{chaum1985security},随后一大批方案相继提出。比较流行的两个方案是IBM的Identity Mixer\citep{camenisch2010specification}和微软的U-Prove方案\citep{paquin2011u}。前者基于CL签名方案\citep{camenisch2001efficient,camenisch2002signature,camenisch2004signature},而后者基于Brands盲签名\citep{brands2000rethinking}。由于CL签名方案建立在RSA群上,Brands盲签名建立在素数阶群上,因此U-Prove方案效率远高于Mixer方案。但是,基于CL签名的是一个可证明安全的方案,而U-Prove方法目前并没有被证明安全。而且,2013年Baldimtsi和Lysyanskaya\citep{baldimtsi2013security}证明了所有已知的随机模型下可证明方法都不能证明Brands盲签名的安全性。同年,他们提出一种可证明安全,且效率值接近U-Prove的方案ACL\citep{baldimtsi2013anonymous}。2014年,Chase等人提出一个基于对称基础设施MAC方案的匿名凭证系统\citep{chase2014algebraic},与U-Prove和ACL相比,除了得到相近的效率外,还能够支持凭证的多次使用而不被链接行为。

直接匿名证明(DAA, Directed Anonymous Attestation)方案是匿名凭证方案的一种应用。第一个DAA协议由\citet{brickell2004direct}提出,基于RSA群上的CL方案实现,随后被TCG采用为TPM 1.2标准中,作为平台身份证明的方案,使得通讯方能够向远程验证方证明自己拥有合法TPM,同时又避免暴露自己的真实身份信息。

随后,DAA方案得到了快速发展,提出了一系列基于ECC的DAA方案\citep{brickell2008new,xiaofeng2008direct,brickell2009simplified,chen2009daa,brickell2010pairing,chen2010design}。ECC-DAA协议的效率比RSA-DAA更高,因此成为了TPM 2.0标准首选的方案。然而由于TPM 1.2种对RSA-DAA协议进行了高度封装实现,即只实现TPM\_Join()和TPM\_Sign()接口,不能灵活实现各类的直接匿名证明协议。因此在TPM 2.0标准中,并没有封装DAA协议接口,而是实现了基础接口,即利用密钥建立接口、承诺接口和签名接口等联合起来实现DAA协议。其中,\citet{chen2010design}和\citet{brickell2010pairing}提出的ECC-DAA方案是目前实现效率较优的方案,被TCG发布的TPM 2.0标准支持。灵活接口的实现使得TPM平台能够支持更多类型的匿名认证协议,比如\citet{chen2013flexible}利用TPM 2.0提供的接口实现了ECC-DAA方案,也给出了对U-Prove方案的支持。

而在DAA协议分析方面,\citet{backes2008zero}首次利用应用Pi演算建模和分析API协议,发现了DAA协议的一个安全缺陷:攻击者可以使得匿名凭证颁发者无法精确统计已持有证书的平台。 \citet{smyth2011formal,Smyth2015Formal}首次利用应用Pi演算对RSA-DAA和ECC-DAA的匿名属性进行定义,并且应用Proverif工具进行安全性分析。 \citet{Brickell2012A}声称大部分DAA方案可以被恶意用户作为静态DH问题预言机,使得安全等级降低为原有的2/3。 \citet{xi2014formal}利用应用Pi演算建立了TPM 2.0中用于实现DAA协议的应用接口模型,定义了匿名性、用户控制的可追踪性和不可陷害性等 安全属性,并利用ProVerif获得安全性分析结果。\citet{Camenisch2016Universally}声称所有已知的DAA安全模型都是不完备或不安全的,并且给出一个在UC框架(Universally Composable Framework)下的安全证明模型。


DAA协议中并没有具体指出如何撤销恶意TPM平台或已被攻克TPM平台的身份证书。一般来说,平台证书的撤销功能通过黑名单实现,当用户身份证书进入黑名单,其便不能认证通过。然而如果直接应用上述匿名凭证系统的机制,会导致匿名性遭到挑战,因为为了实现证书撤销,可信第三方势必要掌握用户的关键身份信息,因此,如果可信第三方与验证者合谋,用户的真实信息就会泄漏。基于此种考虑催生出无可信第三方的匿名凭证系统。

2007年,\citet{brickell2007enhanced}率先扩展了DAA协议,提出了EPID方案,在DAA的基础上增加了证书撤销功能。同年,出现了类似于EPID思想的另一个可撤销的匿名认证系统BLAC\citep{tsang2007blacklistable},这类型的匿名认证系统允许验证者直接撤销恶意用户,而不需要通过可信第三方TTP。以及随后出现效率更高,基于累加器的匿名认证方案PEREA\citep{tsang2008perea}。同时,基于BLAC和PEREA的匿名信誉系统BLACR\citep{au2012blacr}和PERM\citep{au2012perm}也相继被提出。匿名信誉系统的证书撤销功能通过信誉积分来实现,信誉积分有一个阈值,当用户累计的积分超过这个阈值后,用户的证书就会被禁用。与PERM同年提出的基于PEREA的匿名信誉系统PE(AR)$^2$\citep{yu2012pe},取消了PEREA中时间窗口的限制,但赎回分数的方案存在着漏洞。\citet{xi2014arbra}改进了PE(AR)2的漏洞,提出ARBRA方案,但这种设计方案过于复杂。之后,\citet{xi2014farb}设计了新的匿名信誉系统FARB\citep{xi2014farb},兼顾了效率和应用规模,并声称可以运行在资源受限的移动环境中。\citet{henry2013thinking}在2013年通过零知识的批处理技术改进BLACR的效率,提出BLACRONYM方案。2018年由\citet{YangDecentralized}提出DBLACR方案继承BLACR的功能,利用去中心化的思想使得用户的注册过程无需第三方。

TPM 2.0中对DAA协议接口的灵活性也使得TPM 2.0的应用更加广泛,除了在之前提到的由\citet{chen2013flexible}利用TPM 2.0中基础DAA接口实现U-Prove方案。\citet{邵健雄2016下一代可信计算协议的设计与分析}提出的DAA-(AR)$^2$方案利用TPM 2.0中DAA基础接口实现,改进PE(AR)$^2$中的安全问题。这些都说明了TPM 2.0更加广泛的应用前景。

%FAUST方案\citep{lofgren2011faust}基于白名单实现,效率好但功能不完善。

%其他一些基于匿名凭证系统的设计有:电子投票系统\citep{adida2008helios,clarkson2008civitas},电子拍卖系统\citep{mccarthy2014hawk},电子货币系统\citep{chaum1988untraceable,nakamoto2008bitcoin}等。

\section{论文工作} \label{sec:work}

在可信计算平台TPM 2.0规范发布之后,其相对于TPM 1.2进行了很多变化与改进,包括改组授权协议、新增密码学操作原语API以及提升TPM的应用灵活性等。由于在TPM 1.2时期已经有许多分析工作,因此本文重点研究在TPM 2.0中新增和改进的功能模块,以及基于TPM 2.0功能模块所能实现协议的设计与分析研究。本文将研究工作分为三部分:安全协议形式化理论与分析研究、基于TPM 2.0的基础协议和接口的安全性分析研究以及可信计算延伸协议的设计与形式化分析研究。安全协议形式化理论与方法研究是本文的基础工作,是后两个研究工作的基础。而对基础协议和接口的研究可以促进可信计算延伸协议在TPM 2.0的运用。下面我们分别说明三部分的研究内容。

\subsection{安全协议形式化理论与方法研究}

安全协议是保证信息安全的重要方法,对于协议安全性的研究显得尤为重要。对协议的安全性分析的方法之一为形式化分析。形式化分析方案从理论体系来说可以分为三类:模态逻辑技术、模型检测技术和定理证明技术。从敌手模型这个角度来说,可以把形式化分析分为两类:基于符号模型的形式化分析以及基于计算模型的形式化分析。由于本文不研究形式化分析方法的整个理论体系,只研究较为流行的安全协议形式化分析方法。
基于符号模型(Dolev-Yao模型)的协议形式化方法研究
1)	形式化建模语言研究,主要集中于演算类语言研究,包括Pi演算、Spi演算以及由其发展起来的应用Pi演算。而应用Pi演算又作为许多形式化建模语言的基础,研究内容集中于语法语义的研究,包括对其语法扩展后如何保证其语义上的完备性。这种扩展包括增加全局状态的描述、增加无限list的描述以及增加代数性质功能的描述等。这类的研究对实际自动化工具的研究具有理论指导意义。
2)	安全属性的建模研究,包括基本的机密性、对应性,以及更高级的观察等价属性、标记双相似等价属性和迹等价属性等。研究内容包括:重点研究如何利用安全属性来描述协议的安全性质或安全目标,包括匿名认证协议的认证性,匿名性,不可伪造性等;各类安全属性的关系;安全属性对有限消息集合的判定行理论;研究电子投票协议的不可胁迫性的形式化描述等。
3)	自动化工具研究、改进以及应用,这类研究是对上述两种研究的升华,在形式化建模的基础上,对协议的安全属性进行自动化的分析和验证。对自动化工具的研究可以包括三方面:首先,深入研究现有自动化工具的原理和使用方法,利用现有形式化工具对安全协议或接口(比如TPM 2.0 APIs以及匿名认证类协议)的协议原型建模并分析其需要满足的安全目标;其次,在利用现有工具进行建模分析时,对其不能分析的属性和功能进行深入学习,之后改进或扩展这类工具,使之功能更加完整,如Proverif的全局状态支持的扩展;最后,尝试设计和开发形式化工具。目前形式化建模分析方法的研究日趋成熟,如果要从建模理论创新到实现自动化分析工具,难度太大,因此可以从下面两个方面入手进行研究:一是利用应用Pi演算作为高层建模语言,翻译为低层建模语言,并证明转化的可靠性,再利用底层建模语言对应的自动化工具进行分析。比如SAPIC工具(将进程语言转化为Tamarin工具的多重集重写规则);二是利用现有工具作为后端,组合成优势互补的工具,比如可以实现Proverif和CryptoVerif的输入脚本共通性。
基于计算模型的协议形式化方法研究
1)	研究方法之一是采取间接证明的方式,即证明协议在符号模型下的安全性可以保证在计算模型下的可靠性,或基于符号语言的安全模型与基于可证明安全理论的安全模型之间的等价性。从而利用符号模型下自动化工具得到了计算模型下的安全性。
2)	研究方法之二是采取直接证明的方式,即开发自动化工具直接证明协议在计算模型下的安全性,这种方式的理论基础是可证明安全理论中Game序列证明技术,比如CryptoVerif工具和EasyCrypt工具。本课题在计算模型下的形式化分析主要集中在此种方式的研究,研究方向与符号模型大致相似:其一是研究安全协议的建模以安全属性的建模。其二是自动化工具的研究与应用,利用自动化工具对安全协议或API进行建模,并且证明其在计算模型下的安全性。
(2)TPM2.0协议/API的安全性分析与应用
对可信计算的研究关注两个研究点:其一是对可信计算中协议和API接口的形式化建模与分析,从而证明协议或者API组合是否达到预期的安全目标。其二是TPM 2.0技术的推广应用,即利用可信计算技术(功能)构建其他协议与系统。 
1)	密钥管理部分:这部分主要研究形式化分析协议或接口组合的安全性。包括密钥迁移、密钥交换、复制密钥等协议的安全性分析,以及密钥管理API和存储保护类API的安全性分析等;利用形式化工具完成这些协议和接口的分析:符号模型下自动化工具完成协议漏洞分析,计算模型下工具完成协议和接口的计算可靠性保证。
2)	授权协议:除了传统的基于会话的授权,TPM 2.0规范增加了策略授权的功能。对授权部分的研究内容可以为:形式化分析其安全属性,尝试在计算模型下进行形式化分析;根据分析结果,利用授权的功能去设计其他协议或系统架构。
3)	DAA协议:DAA协议是TPM中的一个重要功能,通讯方能够向远程验证方证明自己拥有合法TPM,同时又避免暴露自己的真实身份信息。对DAA功能的研究内容可以分为:一是对协议原型进行形式化建模,分析协议原型的安全性。二是分析TPM中实现DAA功能的API的安全性。可以用形式化方法来分析这类功能接口的安全性。其三可以利用TPM的这类接口来实现其他匿名认证类协议。或者借鉴DAA的硬件保护功能,来增强其他匿名凭证类系统安全性。
4)	移动可信模块:移动可信模块适应资源受限的移动设备和嵌入式环境,许多公司和组织都提出了自家的标准,如MTM,TEE,ARM TrustZone规范等,这些规范都有各自的执行标准。最新提出的TPM 2.0 Mobile参考架构,对各方的架构进行总结,给出了6种参考架构的实现,但是并没有给出具体的功能实现。一般的做法是沿用TPM 2.0命令,但是可以剔除一些不是必要的功能。如何保证剔除一些功能命令后还能保持其安全需求,就可以利用形式化分析的方法,对所需要的安全性质进行建模分析,给出安全使用这些功能的接口执行方式以及参数设定等信息。还可以利用所设计的安全功能,来实现其他应用,比如移动电子支付,移动匿名认证等。
(3)可信匿名凭证系统协议和方案研究
匿名凭证(Anonymous credentials)允许用户进行身份认证,并且保留用户的隐私信息,即允许用户完全控制所要暴露的个人信息,因此达到匿名的效果。目前许多的系统或协议都是基于匿名凭证方案的。比如TPM中的DAA协议,就是基于CL签名的匿名身份认证方法。还有无TTP(可信第三方,Trusted Third Party)的匿名认证系统和匿名信誉系统,以及电子支付和电子票据等。
1)	DAA协议:与2中研究内容相同,不赘述。
2)	无TTP的匿名认证系统和匿名信誉系统:研究这类系统需要利用匿名凭证方案。并且去除了利用第三方权威进行撤销操作。现有这类方案的都有一些未决问题:有的功能丰富但是效率太低;有的效率高但限制某些功能(比如不支持降低信誉);有的构造太过复杂。可研究内容分为四部分:第一,这类系统研究难点在于实现撤销机制与效率之间的冲突,理想的情况是能够实现功能完美且效率高的方案。实际研究可以改进现有的方案或提出新的方案,来实现功能和效率的最优平衡的系统。第二,利用TPM 2.0的DAA功能,实现或改进这类匿名信誉系统,已达到提高安全性的目的。第三,利用形式化分析方法,对这类系统进行建模与安全性分析。形式化分析是服务于第一点,不断进行分析与改进,达到功能与效率的最优。最后,探索在移动和嵌入式这类资源受限的环境下这类系统的设计,注重考虑其效率。
3)	其他匿名认证类协议,包括电子投票和电子拍卖,电子货币和电子金融:电子投票协议难点在于公平性和安全性。公平性即是否被胁迫投票。安全性即在一个不安全的计算环境中完成投票过程。这方面可以研究的点包括:可以利用TPM 2.0接口,实现在非安全环境下构建电子投票系统;完善电子投票协议的安全模型,对相关安全属性进行形式化定义和分析,特别是如何形式化定义抗协迫性。电子货币方案包括了电子票据方案以及近两三年兴起的去中心化货币方案(比Bitcoin和Litecoin)等。这类研究点包括:电子票据需要保证匿名性,但又不能让用户逃票;可以对安全属性建模进行形式化分析,改进现有方案提高安全性。电子货币方案除了提炼其安全模型,改进现有方案外,还可以利用其去中心化的属性,构建一些accountable方案,比如Accountable Storage方案,利用bitcoin强制执行云存储提供商丢失数据的责任(数据丢失则自动转bitcoin给用户)。



\section{论文组织} \label{sec:structure}

本文的后续内容安排如下:

第\ref{chap:mitm}章介绍了双系分析的攻击框架,评估了分组密码Piccolo以及LBlock算法抗双系分析的安全性。除此之外,研制了
一套双系攻击的自动分析软件,用于评估给定分组密码的密钥编排的扩散特性。

第\ref{chap:zc}章改进了已有零相关线性分析的攻击模型,更新了分组密码LBlock以及TWINE算法的安全
性评估结果,以此说明密钥编排中等价密钥对分组密码安全性的影响。

第\ref{chap:diff}章以Zorro算法的差分分析为例,探究了简单密钥编排下分组密码差分分析的适用性。

第\ref{chap:openkey}章主要研究了分组密码在杂凑模式下的安全性,主要包括分组密码在选择密钥假设下的互补性研究以及分组密码
在已知密钥下的扩散性研究。

第\ref{chap:piccolo}章深入分析了Piccolo算法的线性密钥编排,发现其轮常数选取方面存在的安全性隐患;
以此指导简单密钥编排的设计,尤其是轮常数的选取。

第\ref{chap:design}章系统总结了上述几章中提炼出的分组密码密钥编排的设计准则。
并针对一类特殊的密钥编排,给出了较高效的设计流程;并将此设计流程应用于LBlock算法。

第\ref{chap:end}章对全文进行总结,并对后续研究工作进行了简单介绍。
